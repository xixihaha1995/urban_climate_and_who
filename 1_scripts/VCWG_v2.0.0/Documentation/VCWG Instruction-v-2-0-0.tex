%%%%%%%%%%%%%%%%%%%%%%%%%%%%%%%%%%%%%%%%%%%%%%%%%%%%%%%%%%%%%%%%%%%%%%%%%%%
%
% Template for a LaTex article in English.
%
%%%%%%%%%%%%%%%%%%%%%%%%%%%%%%%%%%%%%%%%%%%%%%%%%%%%%%%%%%%%%%%%%%%%%%%%%%%

\documentclass[12pt]{article}

% AMS packages:
\usepackage{amsmath, amsthm, amsfonts}

\usepackage[top=1in, bottom=0.8in, left=0.8in, right=0.8in]{geometry}
\usepackage{graphicx}  
\usepackage{float}
\usepackage{fontenc}
\usepackage{sidecap}
\usepackage{parskip}
\usepackage{siunitx}

\setlength{\parskip}{1em}

\linespread{1}

% Shortcuts.
% One can define new commands to shorten frequently used
% constructions. As an example, this defines the R and Z used
% for the real and integer numbers.
%-----------------------------------------------------------------
\def\RR{\mathbb{R}}
\def\ZZ{\mathbb{Z}}

% Similarly, one can define commands that take arguments. In this
% example we define a command for the absolute value.
% -----------------------------------------------------------------
\newcommand{\abs}[1]{\left\vert#1\right\vert}

% Operators
% New operators must defined as such to have them typeset
% correctly. As an example we define the Jacobian:
% -----------------------------------------------------------------
\DeclareMathOperator{\Jac}{Jac}

%-----------------------------------------------------------------
\title{User's Guide for the Vertical City Weather Generator
	(VCWG v2.0.0)}
\author{Mohsen Moradi and Amir A. Aliabadi\\
  \small This document is typeset using \LaTeX \\
}

\begin{document}
\maketitle

The Vertical City Weather Generator (VCWG) is a software that predicts the urban micro-climate in relation to a nearby rural climate given the urban characteristics. VCWG predicts vertical profiles of temperature, wind speed, humidity, and turbulence kinetic energy as well as the building energy performance metrics in an urban area. More details on the model can be found at the Atmospheric Innovations Research (AIR) laboratory website at www.aaa-scientists.com and corresponding publications \cite{Moradi2021a, Moradi2021b, Aliabadi2021b, Moradi2022}.

\section{Input Dataset}

Note: before publishing forcing datasets, permission should be requested from the data owners. For forcing the model few options are available. For top forcing from a spreadsheet, an excel file is needed from the folder e.g. ``/resources/TopForcing/Vancouver2008\_ERA5\_Jul.csv", which contains latitude, longitude, time difference from UTC, time, air temperature, barometric pressure, incoming longwave, incoming shortwave direct, incoming shortwave diffuse, rain fall, relative humidity, wind direction, and wind speed. This data comes from an ERA5 data product. Inclusion of latent and sensible heat fluxes are not required, but currently they are supplied for July in this dataset for the purpose of evaluating the model. Then VCWG will process this excel file and creates a new EPW formatted file. It learns the format from file ``/resources/epw/rawEPW.epw" and creates the new forcing file as ``/resources/epw/TopForcing.epw".

Alternatively the model can be forced near the surface in a rural area. In this case it is required to put the weather file (*.epw) of the region of interest in the directory e.g. ``/resources/epw/Basel.epw", ``/resources/epw/ERA5\_Basel\_Jun.epw", or ``/resources/epw/ERA5\_Basel.epw". This file can be downloaded from EnergyPlus (https://energyplus.net/) or prepared using alternative datasets. 

In the released version of the software ERA5 datasets are prepared using the ERA5 data product from the European Centre for Medium-Range Weather Forecasts (ECMWF):

(https://www.ecmwf.int/en/forecasts/datasets/reanalysis-datasets/era5).

\section{Input Parameters}

VCWG can take input parameters from the files located in the directory ``/resources/parameters/". These files contain the required parameters of the case study including urban characteristics, vegetation parameters, view factors, simulation parameters, and hydrological parameters. For example, the file ``initialize\_VANCOUVER\_LCZ1.uwg" is associated with running a simulation for 1 month in July. Variable ``OPTION\_RAY'' should be set to 0 to obtain new view factors or otherwise to 1. Variable ``Rural\_Model\_name'' should be set to 1 for MOST or 2 for top forcing. Variable ``EB\_RuralModel\_name'' should be set to 1 for formulation of Louis or 2 for formulation of Penman Monteith in determination of terms in the surface energy balance model in the rural area. Variable ``Tdeep\_ctrl'' should be set to 1 for using the force restore method or 2 for using climate data in forcing the deep soil temperature in the urban area.

If desired, new view factors can be obtained by changing the variable ``OPTION\_RAY" value from 1 to 0. This will create a new file for view factors in e.g. ``ViewFactor\_VANCOUVER\_LCZ1.txt''. Otherwise, the view factors are taken from the same location by a previous run.

\section{Execution}

This version runs VCWG in a single run mode. The single mode only runs the model given one set of input parameters. To run, file ``/VCWG/Run\_VCWGv2.0.0.py" should be executed located in the main directory, the user is required to change the name of ``epwFileName'', ``TopForcingFileName'', ``VCWGParamFileName'', ``ViewFactorFileName'', and ``case''. If top forcing is needed, ``epwFileName'' should be set to ``None''. If top forcing is not needed, ``TopForcingFileName'' should be set to ``None''. The code first checks if there is any ``epwFileName'', if not, then it checks for ``TopForcingFileName''. So an experienced user may still keep the ``TopForcingFileName'' if they are intending for an ``epwFileName'' to be used. File ``ViewFactorFileName'' is generated and over written, if the ``OPTION\_RAY" value is set to 0 in the initialization file. Otherwise, view factors are taken from ``ViewFactorFileName''.

\section{Results}

The outputs of the simulation are going to be saved under the ``Results'' directory. The output is hourly. There are many output files pertaining to climate variables and building performance metrics. It is recommended to discard the first 72 hours (3 days) of simulation as spin-up period while considering results after this period. Writing the simulation results requires the creation of a ``Results'' folder; preferably empty. Otherwise the entire simulation will run and then stop with an error: ``Results folder not found.''

\section{Python Versions}

VCWG is designed to run on Python 3.6.1. This version of Python can be downloaded from the following link ``https://www.python.org/downloads/release/python-361/". For example for a 64-bit Windows operating system the installation file will be ``python-3.6.1-amd64". The following packages and versions can be used: pandas 1.1.4, numpy 1.19.5, scipy 1.1.0, matplotlib 3.1.1. Note that other package versions may also work.

\bibliography{Aliabadi}
\bibliographystyle{apalike}

\end{document}
